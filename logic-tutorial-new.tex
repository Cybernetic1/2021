% TO-DO:
% * fuzzy logic re-write
% * classifying spaces *
% * Yoneda lemma finish
% * functorial semanticss
% * Kripke-Joyal semantics
% * Cohen forcing *
% * Sheaves and topos *
% * Heyting algebra revise *

% * implementation of topology ***
% * model-based approach

% DONE:
% * The model is only for working memory.  
% * Even so, what does the model offer that syntax does not have?
% * Need a simple semantic model to begin thinking....

\input{../YKY-preamble.tex}

\usepackage{color}
\usepackage{mathtools}
\usepackage{hyperref}

\usepackage[backend=biber,style=numeric]{biblatex}
\bibliography{../AGI-book}
% \renewcommand*{\bibfont}{\footnotesize}

\usepackage{graphicx} % Allows including images
\usepackage{tikz-cd}
\usepackage{tikz}
\usepackage[export]{adjustbox}% http://ctan.org/pkg/adjustbox
\usepackage{verbatim} % for comments
% \usepackage{newtxtext,newtxmath}	% Times New Roman font

\numberwithin{equation}{subsection}

\newcommand{\underdash}[1]{%
	\tikz[baseline=(toUnderline.base)]{
		\node[inner sep=1pt,outer sep=10pt] (toUnderline) {#1};
		\draw[dashed] ([yshift=-0pt]toUnderline.south west) -- ([yshift=-0pt]toUnderline.south east);
	}%
}%

\DeclareSymbolFont{symbolsC}{U}{txsyc}{m}{n}
\DeclareMathSymbol{\strictif}{\mathrel}{symbolsC}{74}

\newcommand{\highlight}[1]{\colorbox{pink}{$\displaystyle #1$}}

\newcommand{\emp}[1]{{\color{violet}\textbf{#1}}}
\newcommand*\confoundFace{$\vcenter{\hbox{\includegraphics[scale=0.2]{../confounded-face.jpg}}}$}
\newcommand{\underconst}{\includegraphics[scale=0.5]{UnderConst.png}}
\newcommand{\witness}{\scalebox{0.6}{$\blacksquare$}}
% \newcommand{\Heytingarrow}{\mathrel{-}\mathrel{\triangleright}}
\providecommand\Heytingarrow{\relbar\joinrel\mathrel{\vcenter{\hbox{\scalebox{0.75}{$\rhd$}}}}}

\begin{document}

\title{\cc{\bfseries\color{blue}{\Huge《AGI 逻辑导论》}}
{\bfseries\color{blue}{\Huge《AGI logic tutorial》} }
}
\author{YKY} % Your name
%\institute[] % Your institution as it will appear on the bottom of every slide, may be shorthand to save space
%{
%Independent researcher, Hong Kong \\ % Your institution for the title page
%\medskip
%\textit{generic.intelligence@gmail.com} % Your email address
%}
\date{\today} % Date, can be changed to a custom date

\maketitle

\section*{Summary}
\begin{itemize}
\cc{\item 描述一种可以完整地解决 AGI 的 univeral logic
}{
\item Describes a kind of univeral logic that can completely solve AGI.
}
\end{itemize}

\tableofcontents
% \vspace*{0.5cm}
% 多谢 支持 \smiley

\setcounter{section}{-1}
\section{Background}

\cc{我们想 \textbf{训练} 一个智能系统,训练 是一个 \textbf{机器学习} 的过程,也是一个 \emp{optimization} problem, 目标是将 \textbf{长期的奖励总和} 最大化:
}{
We want to \textbf{training} an intelligent system. Training is a process of \textbf{machine learning} and also a problem of \emp{optimization}. The goal is to maximize \textbf{long-term reward sum}:
}
\begin{equation}
\label{eqn:time-horizon}
\mbox{maximize: } \int_0^{\infty} R \, dt
\end{equation}
\cc{where $R(t)$ = reward at time $t$.  $\int_0^{\infty}$ 表示 计算 累积奖励的 \emp{time horizon}.  (我使用了微分的形式,实际应用通常是离散形式,但两者基本一样,不必深究)
}{
where $R(t)$ = reward at time $t$. $\int_0^{\infty}$ represents the \emp{time horizon} for calculating the cumulative reward. (I used the differential form, the actual application is usually the discrete form , But the two are basically the same, so you don’t have to go into it)
}

\cc{俗语说「棋屎贪吃卒」,在开局初期吃卒,可能导致 $N$步之后被将死,这是\textbf{愚蠢}的行为。 所以 (\ref{eqn:time-horizon})式 令 系统必需顾及长远的利益,遂迫使它学习 \textbf{智慧}。
}{
As the saying goes, "Chess shit is greedy for pawns." It may lead to checkmates after $N$ moves. This is \textbf{stupid} behavior. Therefore, the (\ref{eqn:time-horizon}) formula requires the system to take into account long-term interests, and forces it to learn \textbf{wisdom}.
}

\cc{Architecturally, the AI is a \emp{dynamical system} that constantly updates its ``state'' $\vect{x}$ via: \footnote{Part of the state $\vect{x}$ contains \textbf{sensory input} and \textbf{action output} that allow the AI to interact with the external environment.}
}{
Architecturally, the AI is a \emp{dynamical system} that constantly updates its ``state'' $\vect{x}$ via: \footnote{Part of the state $\vect{x}$ contains \textbf{sensory input} and \textbf{action output} that allow the AI to interact with the external environment.}
}
\begin{equation}
\dot{\vect{x}} = \vect{f}(\vect{x})
\end{equation}
\cc{或者用离散形式表示:
}{
Or expressed in discrete form:
}
\begin{equation}
\vect{x}_{t+1} = \vect{F}(\vect{x}_t)
\end{equation}
\cc{$\vect{F}$ 叫作 transition function.  或者更形象地表示:
}{
$\vect{F}$ is called transition function. Or more vividly:
}
\begin{equation}
\label{eqn:AGI-architecture}
\begin{tikzcd}
\vect{x} \arrow[in=160,out=20,loop,looseness=5,swap,"\vect{F}"]
\end{tikzcd}
\end{equation}
\cc{Our goal is to \textbf{learn} the function $\vect{F}$, implemented as a \emp{deep neural network}.  $\vect{F}$ 包含智能系统内的所有\textbf{知识}。 
}{
Our goal is to \textbf{learn} the function $\vect{F}$, implemented as a \emp{deep neural network}. $\vect{F}$ contains all the \textbf{knowledge} in the intelligent system.
}

\end{document} 
