% TO-DO:

\input{../YKY-preamble.tex}

\usepackage{color}
\usepackage{mathtools}
\usepackage{hyperref}

\usepackage[backend=biber,style=numeric]{biblatex}
\bibliography{../AGI-book}
% \renewcommand*{\bibfont}{\footnotesize}

\usepackage{graphicx} % Allows including images
\usepackage{tikz-cd}
\usepackage{tikz}
\usepackage[export]{adjustbox}% http://ctan.org/pkg/adjustbox
\usepackage{verbatim} % for comments
% \usepackage{tikz-cd}  % commutative diagrams
% \newcommand{\tikzmark}[1]{\tikz[overlay,remember picture] \node (#1) {};}
% \usepackage{booktabs} % Allows the use of \toprule, \midrule and \bottomrule in tables
% \usepackage{amssymb}  % \leftrightharpoons
% \usepackage{wasysym} % frownie face
% \usepackage{newtxtext,newtxmath}	% Times New Roman font
% \usepackage{sansmath}

\numberwithin{equation}{subsection}

\newcommand{\underdash}[1]{%
	\tikz[baseline=(toUnderline.base)]{
		\node[inner sep=1pt,outer sep=10pt] (toUnderline) {#1};
		\draw[dashed] ([yshift=-0pt]toUnderline.south west) -- ([yshift=-0pt]toUnderline.south east);
	}%
}%

\DeclareSymbolFont{symbolsC}{U}{txsyc}{m}{n}
\DeclareMathSymbol{\strictif}{\mathrel}{symbolsC}{74}

\newcommand{\highlight}[1]{\colorbox{pink}{$\displaystyle #1$}}

\newcommand{\emp}[1]{{\color{violet}\textbf{#1}}}
\newcommand*\confoundFace{$\vcenter{\hbox{\includegraphics[scale=0.2]{../confounded-face.jpg}}}$}

\newcommand{\witness}{\scalebox{0.6}{$\blacksquare$}}
% \newcommand{\Heytingarrow}{\mathrel{-}\mathrel{\triangleright}}
\providecommand\Heytingarrow{\relbar\joinrel\mathrel{\vcenter{\hbox{\scalebox{0.75}{$\rhd$}}}}}

\begin{document}

\title{\cc{\bfseries\color{blue}{\Huge《Cover letter -- HKU》}}
{{\Huge《Cover letter -- HKU》} }}
\author{YKY} % Your name
%\institute[] % Your institution as it will appear on the bottom of every slide, may be shorthand to save space
%{
%Independent researcher, Hong Kong \\ % Your institution for the title page
%\medskip
%\textit{generic.intelligence@gmail.com} % Your email address
%}
\date{\today} % Date, can be changed to a custom date

\maketitle

\section{My CV}

\begin{itemize}
	\item In 2019 I discovered that logic structure can be imposed on deep learning by using \textbf{symmetric} neural networks, which emulate the permutation-invariance of logic propositions.
	\item In 2017 I discovered a connection between AI and quantum mechanics: the learning problem in AI is equivalent to solving the Schr\"{o}dinger equation.  But the key idea leading to this insight, ie. the Hamilton-Jacobi-Bellman equation, is already well-known in the literature.

	\item around 2014 I turned towards neural networks for AGI, at the time ``deep learning'' was not yet very popular (ReLU was demonstrated in 2011, Word2Vec was invented in 2013)
	\item 2012 my first and only published paper so far: "Fuzzy-probabilistic logic for common sense", in AGI Conference, Oxford.
	\item from 2004 till 2014 my research focused on classical logic-based AI and I implemented several logic engines
	\item around 2001-2003 I self-taught neuroscience

	\item 我在 GitHub 上有不少项目,包括:
	\begin{itemize}
		\item a few logic engines (in Lisp, Scala, Clojure, etc)
		\item implementation of rete algorithm (cloned from others and improved by me)
		\item genetic algorithm for learning logic rules
		\item simple deep learning experiments (using TensorFlow)
		\item neural network experiments (C++)
		\item a book draft, "Introduction to Strong AI" (Latex)
		\item symmetric neural network tests (TensorFlow \& python code)
	\end{itemize}

	\item 2004 graduated from Hofstra Univ, NY, USA, with BA degree in computer science, chemistry, and English
	\item 1994 majored in Computer Science in CUHK
	\item 我细个12岁时玩电脑已经几叻
	\item 1971 Born
\end{itemize}

% \tableofcontents
% \vspace*{0.5cm}
% 多谢 支持 \smiley

 \setcounter{section}{-1}
\section{Background}

\section*{References}
\cc{欢迎提问和讨论}{Questions, comments welcome} \smiley \\ \vspace*{0.4cm}
% \printbibliography

\end{document} 
